\documentclass[12pt]{article}
\usepackage[lmargin=1in,rmargin=1in,tmargin=1in,bmargin=1in]{geometry}

\def\univname{}
\def\coursenum{}
\def\coursename{Representation Theory of Finite Groups}
\def\professor{}
\def\student{}
\def\semesteryear{Winter 2021}
\def\imagename{iitb.png}		  
\def\scalesize{0.20}
\usepackage{aryaman}

\newcommand{\myemail}{aryaman@iitb.ac.in}

\begin{document}

\section{Introduction}

The topic of study will be Representation Theory of Finite Groups. We shall be using the following notes to study it: \url{https://aryamanmaithani.github.io/math/rep-theory/Notes.pdf}. The aim is to cover Sections 1 and 2. Section 0 can be skipped more or less completely if you are able to solve the questions below. 

The prereqs are Group Theory and Linear Algebra. For the former, if you've done MA 419 or an SoS on Group Theory, you should be okay. For the latter, if you've done MA 401, then very good. Having only done MA 106 is also okay but you should be comfortable with abstract vector spaces. \\
If you're able to solve the questions below within an hour, then you should be comfortable with the content.

I expect you to make a report of your progress in \LaTeX\ as the month goes on. (You should not be starting a report on the last day.)

You \textbf{must} type your solutions in \LaTeX\ and send me a PDF file via email to \href{mailto:\myemail}{\texttt{\myemail}}. If you have any doubt, you may ask it via WhatsApp if you have my number. Otherwise, you can email it to me. The first email that you send to me regarding this should be with the following subject (excluding quotes): ``WoP - Representation Theory''. \\
After that, we can just use that thread for future correspondence, whenever required. \\
I am not going to be too particular about the tiniest of details, your solutions should essentially convince me that you know what you are doing. If you look something up, then mention the source (and only write the solution if \emph{you} understand it!).

There are many tutorials online for learning \LaTeX. One starter is on my website: \url{https://aryamanmaithani.github.io/latex}. You can also find the source files for the earlier linked notes at \url{https://github.com/aryamanmaithani/math/tree/master/rep-theory}.

\section{Questions}

If you wish to use a result from an earlier question, you can do so even if you have not solved the earlier question.

\begin{enumerate}
	\item Let $G$ be a group and $N \unlhd G$ be a normal subgroup. Let $\varphi : G \to H$ be a group homomorphism. We say that \deff{$\varphi$ factors through $G/N$} if there exists a homomorphism $\widetilde{\varphi} : G/N \to H$ such that the following diagram commutes
	\begin{equation} \label{eq:01}
		\begin{tikzcd}
			G \arrow[r, "\varphi"] \arrow[d, "\pi"'] & H \\
			G/N \arrow[ru, "\widetilde{\varphi}"', dashed] & .
		\end{tikzcd}
	\end{equation}
	Show that $\varphi$ factors through $G/N$ iff $N \subset \ker(\varphi)$.

	\textbf{Terminology.} By \Cref{eq:01} commuting, we mean that $\widetilde{\varphi} \circ \pi = \varphi$. As usual, $\pi : G \to G/N$ is the natural map.
	%
	%
	%
	\item Let $G$ be a group. For the elements $g, h \in G$, define the \deff{commutator $[g, h]$} by
	\begin{equation*} 
		[g, h] \vcentcolon= ghg^{-1}h^{-1}.
	\end{equation*} 
	Define $C \subset G$ to be the set of all commutators in $G$, i.e., $C \vcentcolon= \{[g, h] : g, h \in G\}$. The subgroup generated by $C$ is denoted by $[G, G]$. 
	\begin{enumerate}
		\item Show that $[g, h] = 1$ iff $g$ and $h$ commute. 
		\item Show that $[G, G] = \{1\}$ iff $G$ is abelian.
		\item Show that $[G, G] \unlhd G$.
		\item Show that $G/[G, G]$ is abelian.
		\item Let $N \unlhd G$ be such that $G/N$ is abelian. Show that $[G, G] \subset N$. Thus, $[G, G]$ is the smallest normal subgroup we must quotient by, to get an abelian quotient.
		\item Let $\varphi : G \to H$ be a group homomorphism. Show that if $H$ is abelian, then $\varphi$ factors through $G/[G, G]$.
	\end{enumerate}
	%
	%
	%
	%
	\item Let $V$ be a $\mathbb{C}$-vector space. Let $T : V \to V$ be linear. A subspace $W \le V$ is said to be \deff{$T$-invariant} if $T(W) \subset W$, i.e., $T(w) \in W$ for all $w \in W$.
	\begin{enumerate}
		\item Show that $\{0\}$, $V$, $\ker(T)$, and $\im(T)$ are $T$-invariant. ($\ker$ is the kernel, which you may know as null-space. $\im$ is the image.)
		\item Let $S : V \to V$ and $W \le V$ be such that $W$ is both $S$ and $T$-invariant. Show that $W$ is also $S \circ T$-invariant.
		\item Let $S : V \to V$ be such that $T \circ S = S \circ T$. {\color{red}Show that $\ker(S)$ and $\im(S)$ are $T$-invariant.} %\\
		% In words: commuting operators have the same invariant subspaces.
		\item Assume that $T$ is invertible. Show that $W$ is $T$-invariant iff $W$ is $T^{-1}$-invariant.
		\item Let $(V, \langle \cdot, \cdot\rangle)$ be a finite-dimensional inner product space. By $T^{\ast}$, we denote the adjoint of $T$. Suppose that $T$ is unitary and $W$ is $T$-invariant. Show that $W^{\perp}$ is also $T$-invariant.
	\end{enumerate}
	%
	%
	%
	\item Given $\mathbb{C}$-vector spaces $V$ and $W$, define $\Hom_{\mathbb{C}}(V, W)$ to be the set of all $\mathbb{C}$-linear maps from $V$ to $W$.
	\begin{enumerate}
		\item Describe how $\Hom_{\mathbb{C}}(V, W)$ has a natural $\mathbb{C}$-vector space structure.
		\item Assuming $V$ and $W$ to be finite-dimensional, what is the dimension of $\Hom_{\mathbb{C}}(V, W)$?
	\end{enumerate}
	%
	%
	%
	\item From Section 0.1.3. of my notes here: \url{https://aryamanmaithani.github.io/math/rep-theory/Notes.pdfhttps://aryamanmaithani.github.io/math/rep-theory/Notes.pdf#page=11}, look at the concept of linearisation. Prove the following:
	\begin{enumerate}
		\item Let $X$ be a finite set, and $V$ a $\mathbb{C}$-vector space. Let $f : X \to V$ be a function (note that it does not make sense to talk about $f$ being linear). Show that there exists a unique function $F : \mathbb{C}X \to V$ such that $F|_{X} = f$. In other words, the following diagram commutes:
		\begin{equation*} 
			\begin{tikzcd}
				\mathbb{C}X \arrow[r, "F", dashed] & V \\
				X \arrow[u, hook] \arrow[ru, "f"'] & .
			\end{tikzcd}
		\end{equation*}
		(What is the vertical map here?)
		\item Let $X$ and $Y$ be finite sets, and $f : X \to Y$ be a function. Show that there exists a unique function $F : \mathbb{C}X \to \mathbb{C}Y$ such that $F|_{X} = f$. In other words, the following diagram commutes:
		\begin{equation*} 
			\begin{tikzcd}
				\mathbb{C}X \arrow[r, "F", dashed] & \mathbb{C}Y \\
				X \arrow[u, hook] \arrow[r, "f"'] & Y \arrow[u, hook].
			\end{tikzcd}
		\end{equation*}
	\end{enumerate}
	For these questions, it is okay to be brief. Try following the following outline:
	\begin{enumerate}[label=(\roman*)]
		\item Define what $F$ should be, in terms of an equation/formula.
		\item Justify why that $F$ is indeed well-defined.
		\item Mention why that $F$ works.
		\item Mention why that is the only $F$ that works.
	\end{enumerate}
	%
	%
	%
	\item How many conjugacy classes are there in $S_{5}$? Write down exactly one element from each class. \\
	If you're not aware of how conjugacy classes in $S_{n}$ look, you should check Section 0.2.3. of my notes here: \url{https://aryamanmaithani.github.io/math/rep-theory/Notes.pdf#page=17}. Even if you don't want to go through the proof, Theorem 0.39 is something you should know. (The proof is actually simple but notation is daunting. Check out the computations for simple cases to see what's really going on.)
\end{enumerate}
% \coverpage
% \thispagestyle{empty}
% \updated{\today}
% \pagestyle{empty}
% \tableofcontents
% \setcounter{section}{-1}
% \setcounter{page}{1}

% \input{sections/preface}
% \pagestyle{fancy}

\end{document}	