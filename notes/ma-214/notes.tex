\documentclass{article}
\usepackage{amsmath, amssymb, amsfonts, amsthm, mathtools}
\usepackage[utf8]{inputenc}
\usepackage[inline]{enumitem}
\usepackage{cancel}
\usepackage{soul}
\usepackage{hyperref}
\newtheorem{theorem}{Theorem}
\newtheorem{lem}{Lemma}
\newtheorem{defn}{Definition}

\setlength\parindent{0pt}

\usepackage{xcolor}
\definecolor{mybgcolor}{RGB}{50, 50, 50} %46, 51, 63

\usepackage{pagecolor}
\pagecolor{mybgcolor}
\color{white}

\usepackage{geometry}
\geometry{
    a4paper,
    total={170mm,257mm},
    left=20mm,
    top=20mm,
}

\title{MA 214: Numerical Analysis Notes}
\author{Aryaman Maithani}
\date{\today\\
Until: Lecture 12}

\begin{document}
\maketitle
\begin{center}
	\textsc{Disclaimer}
\end{center}
This is just a collection of formulae/algorithms compiled together.\\
In the case of algorithms, I explain the procedure concisely. However, do not take this as a substitute for lecture slides as I don't go into the theory at all.\\
Also, I've modified some things compared to the lecture slides wherever I felt it was an error. So, be warned.\\
Sometimes, I also change the order in the notes compared to how it was taught in slides if I feel like that's more efficient.

\hrulefill

\section{Interpolation}\label{sec:inter}
\begin{enumerate}
	\itemsep1em 
	\item \textbf{Lagrange Polynomials}\\
	Let $x_0,\;x_1,\;\ldots,\;x_{n}$ be $n+1$ distinct points in $[a, b].$ Let $f:[a, b] \to \mathbb{R}$ be a function whose value is known at those aforementioned points. \\
	We want to construct a polynomial $p(x)$ of degree $\le n$ such that $p(x_i) = f(x_i)$ for all $i \in \{0, \ldots, n\}.$\\
	Towards this end, we define the polynomials $I_k(x)$ for $k \in \{0, \ldots, n\}$ in the following manner:
	\[I_k(x) := \prod_{i = 0, i \neq k}^{n}\frac{x - x_i}{x_k - x_i}.\]
	(Intuitive understanding: $I_k$ is a degree $n$ polynomial such that $I_k(x_j) = 0$ if $k \neq j$ and $I_k(x_k) = 1.$)\\
	Now, define $p(x)$ as follows:
	\[p(x) := \sum_{i=0}^{n}f(x_i)I_i(x)\]
	\item \textbf{Newton's form}\\
	Let $x_0, x_1, \ldots, x_n$ be $n+1$ distinct points in $[a, b].$ Let $f:[a, b] \to \mathbb{R}$ be a function whose value is known at those aforementioned points. \\
	We want to construct a polynomial $P_n(x)$ of degree $\le n$ such that $p(x_i) = f(x_i)$ for all $i \in \{0, \ldots, n\}.$\\~\\
	We define the divided differences (recursively) as follows:
	\begin{align*} 
		f[x_0] &:= f(x_0)\\
		f[x_0, x_1, \ldots, x_k] &:= \dfrac{f[x_1, \ldots, x_k] - f[x_0, \ldots, x_{k-1}]}{x_k - x_0} & \text{ for all } 1 < k \le n
	\end{align*}
	With this in place, the desired polynomial $P_n(x)$ is (not so) simply:
	\begin{align*} 
		P_n(x) := f[x_0]  +& f[x_0, x_1](x - x_0)\\
						 +& f[x_0, x_1, x_2](x - x_0)(x - x_1)\\
						 +& \ldots\\
						  &\vdots\\
						 +& f[x_0, x_1, \ldots, x_n](x - x_0)(x - x_1)\cdots(x - x_{n-1})
	\end{align*}
	\emph{Remarks.} Note that $x - x_n$ does not appear in the last term.\\
	Note that given $P_n(x),$ it is simple to construct $P_{n+1}(x).$
	\item \textbf{Osculatory Interpolation}\\
	This is essentially the same as the previous case. \\
	I'll state the problem in the form I think is the simplest. (Any other form can be reduced to this.)\\
	Suppose we are given $k+1$ distinct points $x_0, \ldots, x_k$ in $[a, b]$ and a function $f:[a, b] \to \mathbb{R}$ which is sufficiently differentiable.\\
	Suppose we are given the following values:\\
	\begin{align*} 
		f^{(0)}(x_0), f^{(1)}(x_0)&, \ldots, f^{(m_1-1)}(x_0)\\
		f^{(0)}(x_1), f^{(1)}(x_1)&, \ldots, f^{(m_2-1)}(x_1)\\
		&\vdots\\
		f^{(0)}(x_k), f^{(1)}(x_k)&, \ldots, f^{(m_k-1)}(x_k)\\
	\end{align*}
	(Notation: $f^{(0)}(x) = f(x)$ and $f^{(n)}(x)$ is the $n^{\text{th}}$ derivative.)\\
	Thus, we are given $m_1 + m_2 + \cdots m_k =: n+1$ data. As usual, we now want to compute a polynomial $P_n(x)$ that agrees with $f$ at all the data. (That is, all the given derivatives must also be same.) As it goes without saying, $P_n(x)$ must have degree $\le n.$\\~\\
	To do this, we list the above points as follows:
	\[\underbrace{x_0, x_0, \ldots x_0}_{m_1}, \underbrace{x_1, x_1, \ldots, x_1}_{m_2}, \ldots, \underbrace{x_k, x_k, \ldots, x_k}_{m_k}.\]
	Now, we just apply the above (Newton's) formula with the following modification in the definition of the divided difference:
	\[f[\underbrace{x_i, x_i, \ldots, x_i}_{p+1 \text{ times}}] := \dfrac{f^{(p)}(x_i)}{p!}.\]
	\item \label{rich}\textbf{Richardson Extrapolation}\\
	Suppose that for sufficiently small $h \neq 0,$ we have the formula:
	\[M = N_1(h) + k_1h + k_2h^2 + k_3h^3 + \cdots,\]
	for some constants $k_1, k_2, k_3,\ldots.$\\
	Define the following:
	\[N_j(h) := N_{j-1}(h/2) + \frac{N_{j-1}(h/2) - N_{j-1}(h)}{2^{j-1} - 1} \quad \text{ for } j \ge 2.\]
	Choose some $h$ \emph{sufficiently small} (whatever that means). Then, $N_j(h)$ keeps becoming a better approximation of $M$ as $j$ increases.\\
	We create a table of $h$ and $N_j(h)$ as follows:\\
		\[
		\begin{array}{c|c|c|c|c}
			h & N_1(h) & N_2(h) & N_3(h) & N_4(h)\\
			\hline
			h & N_1(h) & & &\\
			h/2 & N_1(h/2) & N_2(h) & & \\
			h/4 & N_1(h/4) & N_2(h/2) & N_3(h) & \\
			h/8 & N_1(h/8) & N_2(h/4) & N_3(h/2) & N_4(h) \\
		\end{array}
		\]
	$N_4(h)$ will be a good approximation, then.\\
	(Look at slide 15 of Lecture 7 for an example.)\\~\\
	\textbf{Special case}\\
	Sometimes, we may have the following scenario:
	\[M = N_1(h) + k_2h^2 + k_4h^4 + \cdots.\]
	In this case, we define:
	\[N_j(h) := N_{j-1}(h/2) + \frac{N_{j-1}(h/2) - N_{j-1}(h)}{4^{j-1} - 1} \quad \text{ for } j \ge 2.\]
	Then, we do the remaining stuff as before.
\end{enumerate}

\section{Numerical Integration}
\[I = \int_{a}^{b} f(x) \text{d}x\]
\begin{enumerate} 
	\itemsep1em
	\item \textbf{Rectangle Rule}
		\[I \approx (b - a)f(a)\]
		\[E^R = f'(\eta)\frac{(b-a)^2}{2},\;\text{for some }\eta\in[a, b]\]
	
	\item \textbf{Midpoint Rule}
		\[I \approx (b - a)f\left(\frac{a + b}{2}\right) \]
		\[E^M = \frac{f''(\eta)}{24}(b - a)^3,\;\text{for some }\eta\in[a, b]\]
	
	\item \textbf{Trapezoidal Rule}
	\[I \approx \frac{1}{2}(b - a)[f(a) + f(b)]\]
	\[E^T = -f''(\eta)\frac{(b - a)^3}{12},\;\text{for some }\eta\in[a, b]\]

	\item \textbf{Corrected Trapezoidal}
	\[I \approx \frac{1}{2}(b - a)[f(a) + f(b)] + \frac{(b - a)^2}{12}(f'(a) - f'(b))\]
	\[E^{CT} = f^{(4)}(\eta)\frac{(b - a)^5}{720},\;\text{for some }\eta\in[a, b]\]
	\item \textbf{Composite Trapezoidal}
	\[I \approx \frac{h}{2}\left[f(x_0) + 2\sum_{i=1}^{N-1}f(x_i) + f(x_N)\right]\]
	\[E_C^{T} = -f''(\xi)\frac{h^2(b-a)}{12},\;\text{for some }\xi\in[a, b]\]
	Here, $Nh = b-a$ and $x_i = a + ih.$\\
	\item \textbf{Simpson's Rule}
	\[I \approx \frac{b - a}{6}\left\{f(a) + 4f\left(\frac{a + b}{2}\right) + f(b)\right\}\]
	\[E^S = -\frac{1}{90}f^{(4)}(\eta)\left(\frac{b-a}{2}\right)^5,\;\text{for some }\eta\in[a, b]\]
	\item \textbf{Composite Simpson's}
	\[I \approx \frac{h}{6}[f(x_0) + 2\sum_{i=1}^{N-1}f(x_i) + 4\sum_{i=1}^{N}f\left(x_{i-1} + \frac{h}{2}\right) + f(x_N)]\]
	\[E_C^{S} = -f^{(4)}(\xi)\frac{(h/2)^4(b-a)}{180},\;\text{for some }\xi\in[a, b]\]
	Here, $Nh = b-a$ and $x_i = a + ih.$\\
	\item \textbf{Gaussian Quadrature}\\
	Let $Q_{n+1}(x)$ denote the $(n+1)^{\text{th}}$ Legendre polynomial.\\
	Let $r_0, \ldots, r_{n+1}$ be its roots. (These will be distinct, symmetric about the origin and will lie in the interval $[-1, 1].$)\\
	For each $i \in \{0, \ldots, n\},$ we define $c_i$ as follows:
	\[c_i := \int_{-1}^{1} \left(\prod_{k = 0, k \neq i}^{n}\frac{x - x_k}{x_i - x_k}\right) \text{d}x.\]
	Then, we have
	\[\int_{-1}^{1} f(x) \text{d}x \approx \displaystyle\sum_{i=0}^{n}f(r_i)c_i.\]
	Moreover, if $f$ is a polynomial of degree $\le 2n+1,$ then the above is ``approximation'' is exact.\\~\\
	Standard values: \\
	$n = 0:$ $Q_1(x) = x$ and $x_0 = 0.$ $c_0 = 2.$\\
	$n = 1:$ $Q_2(x) = x^2 - \frac{1}{3}$ and $x_0 = -\frac{1}{\sqrt{3}},\;x_1 = \frac{1}{\sqrt{3}}.$ $c_0 = c_1 = 1.$\\
	$n = 2:$ $Q_3(x) = x^3 - \frac{3}{5}x$ and $x_0 = -\sqrt{\frac{3}{5}},\;x_1 = 0,\; x_2 = \sqrt{\frac{3}{5}}.$ $c_0 = c_2 = 5/9,\;c_1 = 8/9.$
	\item \textbf{Improper integrals using Taylor series method}\\
	Suppose we have $f(x) = \frac{g(x)}{(x - a)^p}$ for some $0 < p < 1$ and are asked to calculate $I = \displaystyle\int_{a}^{b} f(x) \text{d}x.$\\
	For the sake of simplicity, I assume $a = 0$ and $b = 1.$\\
	Let $P_4(x)$ denote the fourth Taylor polynomial of $g$ around $a.$ (In this case $0$.)\\
	Now, compute $I_1 = \displaystyle\int_{0}^{1} \frac{P_4(x)}{x^p} \text{d}x.$ This can be integrated exactly. \hfill (Why?)\\
	Now, we approximate $I - I_1.$\\
	Define
	\[G(x) := \left\{\begin{array}{ll}
		\frac{g(x) - P_4(x)}{x^p} & \text{ if } 0 < x \le 1\\
		0 & \text{ if } x = 0
	\end{array}
	\right.\]
	Then, approximate $I_2 = \displaystyle\int_{0}^{1} G(x) \text{d}x$ using the composite Simpson's rule.\\
	Then, $I = I_1 + I_2.$\\~\\
	For the case of $a = 0,\; b = 1$ and $N = 2$ for the composite Simpson's part, we get that \\
	$I_2 \approx \frac{1}{12}[2G(0.5) + 4G(0.25) + 4G(0.75) + G(1)].$\\~\\
	That is, finally:
	\[I \approx \int_{0}^{1} \frac{P_4(x)}{x^p} \text{d}x + \frac{1}{12}[2G(0.5) + 4G(0.25) + 4G(0.75) + G(1)].\]
	%
	\item \textbf{Adaptive Quadrature}\\
	Let $I = \displaystyle\int_{a}^{b} f(x) \text{d}x$ be the integral that we want to approximate.\\
	Suppose that $\epsilon$ is the accuracy to which we want $I.$ That is, we want a number $P$ such that $|P - I| < \epsilon.$\\
	Here is what you do:\\
	Subdivide $[a, b]$ into $N$ intervals: $[x_0, x_1],\;[x_1, x_2],\;\ldots,\;[x_{n-1}, x_n].$ (Naturally, $a = x_0 \le x_1 \le \ldots \le x_n = b$.)\\
	Now, for each subinterval, compute the following values:\\~\\
	$S_i = \dfrac{h}{6}\left(f(x_i) + 4f\left(x_i + \dfrac{h}{2}\right) + f\left(x_{i+1}\right)\right),$ and\\~\\
	$\overline{S_i} = \dfrac{h}{12}\left(f(x_i) + 4f\left(x_i + \dfrac{h}{2}\right) + 2f\left(x_i + \dfrac{h}{2}\right) + 4f\left(x_i + \dfrac{3h}{4}\right) + f(x_{i+1})\right).$\\~\\
	Now, calculate $E_i = \frac{1}{15}|\overline{S_i} - S_i|.$\\~\\
	Now, if $E_i \le \frac{x_{i} - x_{i-1}}{b - a}\epsilon,$ then move on to the next interval.\\
	Otherwise, subdivide again to better approximate $\displaystyle\int_{x_{i-1}}^{x_i} f(x) \text{d}x.$\\~\\
	Finally, sum up all the $\overline{S_i}$s and that's the answer. That is,
	\[I \approx P = \sum_{i=1}^{n}\overline{S_i}.\]
	%
	\item \textbf{Romberg Integration}\\
	Essentially the baby of composite Trapezoidal rule and Romberg integration.\\
	Suppose we want to calculate $\displaystyle\int_{a}^{b} f(x) \text{d}x.$\\
	Let $N$ be a power of $2$.\\
	$T_N := \dfrac{h}{2}\left[f(x_0) + 2\displaystyle\sum_{i=1}^{N-1}f(a + ih) + f(x_N)\right],$ where $Nh = b-a.$\\~\\
	Note that $T_N$ can be computed using $T_{N/2}$ (assuming $N \neq 2^0$) as:
	\[T_N = \frac{T_{N/2}}{2} + h\sum_{i=1}^{N/2}f\left(a + (2i - 1)h\right).\]
	Now for $m \ge 1,$ we define:
	\[T^{m}_N = T^{m-1}_N + \frac{T^{m-1}_N - T^{m-1}_{N/2}}{4^m - 1}.\]
	(Where $T^{0}_N$ is just $T_N.$)\\
	(Also, for some reason, $T'_N$ has been used instead of $T^1_N$.)\\
	Note that $\frac{N}{2^m}$ must be an integer for $T^m_N$ to be defined.
	We create a table as follows:\\
		\[
		\begin{array}{c|c|c|c|c}
			N & T_N & T'_N & T^2_N & T^3_N\\
			\hline
			1 & T_1 & & &\\
			2 & T_2 & T^1_2 & & \\
			4 & T_4 & T^1_4 & T^2_4 & \\
			8 & T_8 & T^1_8 & T^2_8 & T^3_8 \\
		\end{array}
		\]
	$T^3_8$ will be a good approximation, then.\\
	(Look at slide 25 of Lecture 7 for an example.)\\~\\
	\emph{Remark.} It can be shown that $I = T_N + c_2h^2 + c_4h^4 + \cdots.$ This is why we used the special case formula of \ref{sec:inter}. \ref{rich}.
\end{enumerate}

\section{Numerical Differentiation}
\begin{enumerate} 
	\itemsep1em
	\item \[f'(a) \approx \frac{f(a + h) - f(a)}{h}\]
	\[E(f) = -\frac{1}{2}hf''(\eta) \quad \text{ for some }\eta\in[a, a+h].\]
	\item \textbf{Central Difference Formula}\\
	\[f'(a) \approx \frac{f(a + h) - f(a-h)}{2h}\]
	\[E(f) = -\frac{1}{6}h^3f^{(3)}(\eta) \quad \text{ for some }\eta\in[a-h, a+h].\]
	Note that this is an $O(h^2)$ approximation. Thus, we can use the special case of \S\ref{sec:inter}. \ref{rich}. for better accuracy.
	\item \[f'(a) \approx \frac{-3f(a) + 4f(a + h) - f(a + 2h)}{2h}\]
	\[E(f) = \frac{1}{3}h^3f^{(3)}(\eta) \quad \text{ for some }\eta\in[a, a+2h].\]\\
	\\
	Formula 2 is always the better one whenever applicable. At end points, formula 3 is better than formula 1.
	\item \textbf{Central difference for second derivative}\\
	\[f''(x_0) = \frac{f(x_0 + h) - 2f(x_0) + f(x_0 - h)}{h^2} - \frac{h^2}{12}f^{(4)}(\xi),\]
	for some $\xi \in (x_0 - h, x_0 + h).$
	%
	\item \textbf{Solving boundary-value problems in ODE}\\
	Suppose that we want to solve the following (linear) ODE:
	\[y''(x) + f(x)y'(x) + g(x)y = q(x)\]
	in the interval $[a, b]$ such that we know $y(a) = \alpha,$ and $y(b) = \beta.$ \\~\\
	Set $h := \frac{b - a}{N}$ for some $N \in \mathbb{N}$ and $x_0 = a + ih$ for $h \in \{0, 1, \ldots, N\}.$\\
	Using central difference approximation, we set up $N-1$ linear equations as follows:
	\[\frac{y_{i-1} - 2y_0 + y_i}{h^2} + f(x_i)\frac{y_{i+1} - y_{i-1}}{2h} + g(x_i)(y_i) = q(x_i)\]
	\[i = 1, 2, \ldots, N-1\]
	The above equations can be rearranged as:
	\[\left(1 - \frac{hf_i}{2}\right)y_{i-1} + (-2 + h^2g_i)y_i + \left(1 + \frac{hf_i}{2}\right)y_{i+1} = h^2q_i,\]
	for $i = 1, \ldots, N-1;$ where $f_i = f(x_i)$ and so on.\\
\end{enumerate}

\section{Solution of non-linear equations}
Let $f$ be a continuous function on $[a_0, b_0]$ such that $f(a_0)f(b_0) < 0$ in all these cases. We want to find a root of $f$ in $[a_0, b_0].$ (Existence in implied.)
\begin{enumerate} 
	\itemsep1em
	\item \textbf{Bisection Method}\\
	Set $n = 0$ to start with.\\
	Loop over the following:\\
	Set $m = \frac{a_n + b_n}{2}.$\\
	If $f(a_n)f(m) < 0,$ then set $a_{n+1} = a_n$ and $b_{n+1} = m.$\\
	Else, set $a_{n+1} = m$ and $b_{n+1} = b_n.$\\
	Increase $n$ by one.\\
	We still have a root in $[a_n, b_n].$
	%
	\item \textbf{Regula-falsi or false-position method}\\
	Set $n = 0$ to start with.\\
	Loop over the following:\\
	Set $w = \frac{f(b_n)a_n - f(a_n)b_n}{f(b_n) - f(a_n)}.$\\
	If $f(a_n)f(w) < 0,$ then set $a_{n+1} = a_n$ and $b_{n+1} = w.$\\
	Else, set $a_{n+1} = w$ and $b_{n+1} = b_n.$\\
	Increase $n$ by one.\\
	We still have a root in $[a_n, b_n].$
	%
	\item \textbf{Modified regula-falsi}\\
	Set $n = 0$ and $w_0 = a_0$ to start with.\\
	Loop over the following:\\
	Set $F = f(a_n)$ and $G = f(b_n).$\\
	Set $w_{n+1} = \dfrac{Ga_n -Fb_n}{G - F}.$\\
	If $f(a_n)f(w_{n+1}) \le 0,$ then set $a_{n+1} = a_n$ and $b_{n+1} = w_{n+1}$ and $G =f(w_{n+1}).$\\
	Furthermore, if we also have $f(w_{n})f(w_{n+1}) > 0,$ set $F = \frac{F}{2}.$\\
	Else, set $a_{n+1} = w_{n+1}$ and $b_{n+1} = b_n$ and $F = f(w_{n+1}).$\\
	Furthermore, if we also have $f(w_{n})f(w_{n+1}) > 0,$ set $G = \frac{G}{2}.$\\
	Increase $n$ by one.\\
	We still have a root in $[a_n, b_n].$	
	%
	\item \textbf{Secant method}\\
	Set $x_0 = a,$ $x_1 = b$ and until satisfied, keep computing $x_n$ given by
	\[x_{n+1} = \frac{f(x_n)x_{n-1} - f(x_{n-1})x_n}{f(x_n) - f(x_{n-1})} \qquad \text{ for } n \ge 1.\]
	\emph{Remark.} This process will be forced to stop if we arrive at $f(x_n) = f(x_{n-1})$ at some point.
\end{enumerate}

\section{Iterative methods}
\begin{enumerate} 
	\itemsep1em
	\item \textbf{Newton's Method} \\
	You are given a function $f$ which is continuously differentiable and you want to find its root. You are also given some $x_0.$\\
	Compute the following sequence recursively until satisfied:
	\[x_{n+1} = x_n - \frac{f(x_n)}{f'(x_n)} \qquad \text{ for } n \ge 0.\]
	%
	\item \textbf{Fixed point iteration}\\
	Let $I$ be a closed interval in $\mathbb{R}.$ Let $f:I\to I$ be a differentiable function such that there exists some $K \in [0, 1)$ such that $|f'(x)| \le K$ for all $x \in I.$\\
	Then, there is a unique $\xi \in I$ such that $f(\xi) = \xi.$ To find this fixed point, choose any $x_0 \in I$ and define the sequence
	\[x_n := f(x_{n-1}) \quad n \ge 1.\]
	Then, $x_n \to \xi.$
	\item \textbf{Aitken's} $\mathbf{\Delta^2}$ \textbf{Process}\\
	\textbf{Definition.} Given a sequence $(x_n),$ let $\Delta x_n := x_{n+1} - x_n.$\\
	Then, $\Delta^2x_n = x_{n+2} - 2x_{n+1} + x_n.$\\~\\
	
	Given a sequence $x_0, x_1, \ldots$ converging to $\xi,$ calculate $\widehat{x_1}, \widehat{x_2}, \ldots$ by
	\[\widehat{x_n} := x_{n+1} - \frac{(\Delta x_n)^2}{\Delta^2x_{n-1}}.\]
	Then, $\widehat{x_n} \to \xi.$\\~\\
	%
	If the sequence $x_0, x_1, \ldots$ converges linearly to $\xi,$ that is, if
	\[\xi - x_{n+1} = K(\xi - x_n) + \theta(\xi - x_n), \qquad \text{ for some } K \neq 0\]
	then $\widehat{x_n} = \xi + O(\xi - x_n),$ that is, $\dfrac{\widehat{x_n} - \xi}{x_n - \xi} \to 0.$
	\item \textbf{Steffensen iteration}\\
	Let $g(x)$ be the function whose fixed point is desired. Let $y_0$ be some given point.\\
	Set $n = 0$ to start with.\\
	Loop over the following:\\
	Set $x_0 = y_n.$\\
	Set $x_1 = g(x_0),\;x_2 = g(x_1).$\\
	Set $\Delta x_1$ and $\Delta^2x_0.$\\~\\
	Set $y_{n+1} = x_2 - \dfrac{(\Delta x_1)^2}{\Delta^2x_0}.$\\~\\
	Increase $n$ by 1.\\~\\
	Note that we get a sequence $y_0, y_1, y_2, \ldots.$ However, we only ever have $x_0, x_1$ and $x_2.$
\end{enumerate}

\begin{defn} 
	Let $x_0, x_1, x_2, \ldots$ be a sequence that converges to $\xi$ and set $e_n = \xi - x_n.$\\
	If there exists a number $P$ and a constant $C \neq 0$ such that
	\[\lim_{n\to \infty}\frac{|e_{n+1}|}{|e_n|^P} = C,\]
	then $P$ is called the \textbf{order of convergence} and $C$ is called \textbf{asymptotic error constant}.
\end{defn}

\textbf{Examples.}
\begin{enumerate} 
	\item \textbf{Fixed point iteration}\\
	$\xi$ fixed point of $g:I\to I$ and $g'(\xi) \neq 0.$\\
	$P = 1$ and $C = |g'(\xi)|.$
	\item \textbf{Newton's method}\\
	$\displaystyle\lim_{n\to \infty}\frac{|e_{n+1}|}{|e_n|} = \frac{1}{2}\left|\frac{f''(\xi)}{f'(\xi)}\right|.$\\
	(If $\xi$ is a double root, then $P = 1.$)
	\item \textbf{Secant method}
	\[|e_{n+1}| = C|e_n||e_{n-1}|\]
	$P = \frac{1 + \sqrt{5}}{2} = 1.618\ldots.$\\
	\[\lim_{n\to \infty}\frac{|e_{n+1}|}{|e_n|^P} = \left|\frac{1}{2}\frac{f''(\xi)}{f'(\xi)}\right|^{1/P},\;\text{ provided }f'(\xi) \neq 0.\]
\end{enumerate}

\begin{theorem} 
	Let $f:[a, b] \to \mathbb{R}$ be in $C^2[a, b]$ and let the following conditions be satisfied:
	\begin{enumerate}[nosep] 
		\item $f(a)f(b) < 0,$
		\item $f'(x) \neq 0,$ for all $x \in [a, b],$
		\item $f''(x)$ doesn't change sign in $[a, b]$ (might be zero at some points),
		\item 
		\[\frac{|f(a)|}{|f'(a)|} \le b - a \text{ and }\frac{|f(b)|}{|f'(b)|} \le b - a.\]
	\end{enumerate}
	Then, the Newton's method converges to the unique solution $\xi$ of $f(x) = 0$ in $[a, b]$ for any choice $x_0 \in [a, b].$
\end{theorem}

\section{Solving systems of linear equations}
\subsection{LU Factorisation}

	We want solve $Ax = b$ where $A$ is some known $n\times n$ matrix, $b$ a known $n\times1$ matrix and $x$ is unknown.\\
	\emph{Assumption}: $Ax = b$ can be solved without any row interchange.\\
	We define (finite) sequences of matrices $A^{(n)} = [a^{(n)}_{ij}]$ and $b^{(n)}.$ \\
	Define $A^{(1)} := A.$ Let $m_{ji} := \dfrac{a^{(1)}_{ji}}{a^{(i)}_{ii}}.$\\~\\
	Define $M^{(1)}$ as
	\[M^{(1)} := 
	\begin{bmatrix}
		1 & 0 & 0 & \cdots & 0\\
		-m_{21} & 1 & 0 & \cdots & 0\\
		-m_{31} & 0 & 1 & \cdots & 0\\
		\vdots & \vdots & \vdots & \ddots & \vdots\\
		-m_{n-1, 1} & 0 & 0 & \cdots & 0\\
		-m_{n1} & 0 & 0 & \cdots & 1\\
	\end{bmatrix}\]
	Thus, we can write $M^{(1)}A^{(1)}x = M^{(1)}b.$\\
	Let $A^{(2)} := M^{(1)}A^{(1)}$ and $b^{(2)} = M^{(1)}b^{(1)}.$\\
	Note that $A^{(2)}$ will be a matrix identical to the first one with respect to the last $n-1$ columns. However, it's first column will just have the top element non-zero and everything below will be zero.\\~\\
	We can similarly construct the later matrices that perform the row operations. In general, we have:
	\[M^{(k)} := 
	\begin{bmatrix}
		1 & 0 & \cdots & 0 & \cdots & 0\\
		0 & 1 & \cdots & 0 & \cdots & 0\\
		\vdots & \vdots & \vdots & \vdots & \ddots & \vdots\\
		0 & 0 & \cdots & 1 & \cdots & 0\\
		0 & 0 & \cdots & -m_{k+1, k} & \cdots & 0\\
		\vdots & \vdots & \vdots & \vdots & \ddots & \vdots\\
		0 & 0 & \cdots & -m_{n, k} & \cdots & 1\\
	\end{bmatrix},\]
along with
\[A^{(k+1)} = M^{(k)}A^{(k)} = M^{(k)}\cdots M^{(1)}A, \text{ and}\]
\[b^{(k+1)} = M^{(k)}b^{(k)} = M^{(k)}\cdots M^{(1)}b.\]
Finally, set $U = A^{(n)}$ and $L = [M^{(1)}]^{-1}\cdots \left[M^{(n-1)}\right]^{-1}.$\\
Then, we have
\[L = 
\begin{bmatrix}
	1 & 0 & 0 & \cdots & 0\\
	m_{21} & 1 & 0 & \cdots & 0\\
	m_{31} & m_{32} & 1 & \cdots & 0\\
	\vdots & \vdots & \vdots & \ddots & \vdots\\
	m_{n-1, 1} & m_{n-1, 2} & m_{n-1, 3} & \cdots & 0\\
	m_{n1} & m_{n2} & m_{n3} & \cdots 1
\end{bmatrix}.\] 
Thus, we have $A = LU.$ Now, set $y = Ux.$ We solve $Ly = b$ for $y.$ This is easy because $L$ is lower triangular.\\
Then, we solve $Ux = y$ for $x.$\\~\\
Check slide 27 of Lecture 11 for example.\\~\\
LU factorisation requires $\theta(n^3/3)$ multiplication and division and $\theta(n^3/3)$ addition and subtraction. However, once the factorisation is done, it can be used again and again to solve $Ax = b$ for different values of $b.$ The number of equations then taken to solve $Ax = b$ is $\theta(2n^2).$

\subsection{LPU Factorisation}
In the previous case, we assumed that we aren't doing any row operations on $A$ when doing the Gauss elimination. Now, we relax that condition.\\
Suppose, if possible, we have done some row exchanges while doing Gauss elimination on $A.$ Construct the matrix $P$ which is obtained upon doing those exact row exchanges on $I.$ (Only the row exchanges.)\\
Suppose that the the final product of doing the Gauss elimination following the previous procedure gave us the upper triangular matrix as $U$ and lower one as $L.$\\
Then, that means $PA = LU.$ Thus, solving $Ax = b$ can first be reduced to $PAx = Pb = b'.$ Now, converting $PA$ to $LU$ does not take any row exchange, so we're back in business. (Back to the previous case, that is.)\\
Check slide 15 of Lecture 12 for example.

\subsection{Cholesky's Algorithm}
\begin{defn} 
	A matrix $A$ is positive definite if:
	\begin{enumerate} 
		\item $A = A^t,$ and
		\item $x^tAx > 0$ for all $x \neq 0.$
	\end{enumerate}
\end{defn}
Given any positive definite matrix $A,$ we can write $A = LL^t,$ where $L$ is lower triangular. (And thus, $L^T$ would be upper triangular.)\\
The algorithm to do so is as follows:\\
Let $L = (l_{ij}).$ We now construct these entries.\\~\\
\texttt{
	set $l_{ii} := \sqrt{a_{11}}.$\\
	for $j = 2, 3, \ldots, n:$ \\
	\phantom{123456} set $l_{j1} := \frac{a_{j1}}{l_{11}}.$\\
	for $i = 2, 3, \ldots, n - 1:$\\~\\
	\phantom{123456} set $l_{ii} := \sqrt{\displaystyle a_{ii} - \sum_{k=1}^{i-1}l_{ik}^2}$\\~\\
	\phantom{123456} for $j = i+1, \ldots, n:$ \\
	\phantom{123456}\phantom{123456}set $l_{ji} := \displaystyle a_{ji} - \frac{\displaystyle\sum_{k=1}^{i-1}l_{jk}l_{ik}}{l_{ii}}.$\\~\\
	set $l_{nn} := \sqrt{a_{nn} - \displaystyle\sum_{k=1}^{n-1}l_{nk}^2}.$
}\\~\\
Note that in the above, we've only defined $l_{ij}$ for when $i \ge j.$ The others are obviously $0.$\\~\\
Cholevsky's algorithm requires $\theta(n^3/6)$ multiplication and division and $\theta(n^3/6)$ addition and subtraction.

\subsection{Scaled partial pivoting}
Suppose we want to solve $Ax = b.$ This could lead to round off errors (check slide 17 of Lecture 12). One possible way to combat this is to do the following:\\~\\
First, define the scale factor $s_i$ of row $i$ as the maximum modulus of any element in the row. More precisely, $s_i := \displaystyle\max_{1 \le j \le n} |a_{ij}|.$\\
(Note that we assume that $A$ is invertible and thus, every $s_i$ will be nonzero.)\\~\\
Now, along the first column, we find the row which has the greatest ratio $\dfrac{|a_{k1}|}{s_k}.$ Suppose this is achieved for row $p,$ that is,
\[\frac{|a_{p1}|}{s_p} = \max_{1 \le k \le n}\frac{|a_{k1}|}{s_k}.\]
If $p \neq 1,$ then we perform $R_1 \leftrightarrow R_p.$\\
(Now, when we refer to $s_p$, we will mean the scale factor of the new $p,$ that is, the original $s_1.$ Similarly, for $s_1.$)\\
Now, we make everything below $a_{11}$ zero using the standard row operation.\\~\\
Now, we repeat the same thing by going along column 2 and finding the row which has the greatest ratio $\dfrac{|a_{k2}|}{s_k}.$ If the row is not 2, then we perform the interchange and go on.\\~\\
Two things I would like to point out:\\
1. We never change the value of $s_i$ except the interchanging that we do. In particular, after doing the row operations like $R_2 - m_{21}R_1,$ I will \textbf{not} recalculate the value of $s_2.$\\
2. When considering the ratios $\dfrac{|a_{ki}|}{s_k}$ for column $i,$ we will consider the $a_{ki}$ that is currently present. That is, in this case, we will consider the values that we get after performing all the operations that have been performed until that point.\\~\\
Consider the example given in slide 22 of Lecture 12. After doing $R_1 \leftrightarrow R_3, R_2 - \frac{4.01}{1.09}R_1,$ and $R_3 - \frac{2.11}{1.09}R_1,$ we \emph{do not} recalculate $s_i.$ (We do exchange the values.)\\
However, when considering the ratios, we consider the new matrix for taking $a_{k2}$ obtained after the above subtractions.
\end{document}