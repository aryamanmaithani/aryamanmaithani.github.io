\sec{The Method of Undetermined Coefficients \label{sec:undcoeff}}
\subsection{The first challenge} \label{ssec:undcoeffhomo}
\begin{mdframed}[style=boxstyle, frametitle={The Setup}]
	Consider a linear ODE of the form
	\begin{equation}\label{eq:const}
		y^{(n)} + a_{n-1}y^{(n - 1)} + \cdots + a_0y = 0,
	\end{equation}
	where $a_0, \ldots, a_{n-1}$ are constants. (Real or complex.)\\
	This is a \defin{constant coefficients ODE}.
\end{mdframed}
\begin{mdframed}[style=boxstyle, frametitle={Solving it}]
	We plug the trial solution $y = e^{mx}.$ This gives us the polynomial equation:
	\begin{equation}\label{eq:char}
		m^n + a_{n-1}m^{n-1} + \cdots + a_0 = 0
	\end{equation}
	If $m_1, \ldots, m_n$ are \emph{distinct} solutions to the above equation, then $e^{m_1x}, \ldots, e^{m_nx}$ are $n$ linearly independent solutions of the ODE and thus, we are done.
\end{mdframed}
\begin{mdframed}[style=boxstyle, frametitle={Repeated roots}]
	Suppose $m_0$ is a repeated root of (\ref{eq:char}). Suppose that it is repeated $k$ times. Then, one can show that the following $k$ functions are solutions of the original ODE:
	\begin{equation*} 
		e^{m_0x}, xe^{m_0x}, \ldots, x^{k-1}e^{m_0x}.
	\end{equation*}
	These are linearly independent as well.\\
	Thus, we are now done for any case as the total number of roots is always going to be $n$ when counted with multiplicity. (This is assuming that we work in $\mathbb{C},$ which we shall do.)
\end{mdframed}
\begin{mdframed}[style=boxstyle, frametitle={Getting real}]
	Suppose we are in the case where each $a_i$ is real.\\
	In this case, if $m = a + ib$ is a solution, then so is $m' = a - ib.$ Moreover, the ``amount of repetition'' will also be same.\\
	Thus, we can replace solutions having $\{e^{(a + ib)x}, e^{(a - ib)x}\}$ with \\
	$\{e^{ax}\sin bx, e^{ax}\cos bx\}.$\\~\\
	Similarly considerations apply to functions like $x^ke^{(a + ib)x}.$
\end{mdframed}
\exercise{%
Solve the following ODEs.
\begin{enumerate}[leftmargin=*]
	\item $y'' - 3y' + 2y = 0.$
	\item $y^{(4)} + y = 0.$
	\item $y^{(4)} + y^{(2)} + y = 0.$
\end{enumerate}}
\subsection{Annihilators}
Before going into the main problem, let us study annihilators. \\
In the following we shall use the following notation:
\begin{mdframed}[style=boxstyle, frametitle={Notation}]
	From here on, we shall write $D$ to mean $\dfrac{d}{dx}.$\\
	Similarly, we have $D^n = \dfrac{d^n}{dx^n}.$ \\~\\
	Note that $D$ is an ``operator'' which ``acts'' on smooth\footnote{Recall that a smooth function is a function that is infinitely differentiable.} functions to give another smooth function.
\end{mdframed}
	
\begin{mdframed}[style=boxstyle, frametitle={Some arithmetic}]
	The operator follows the usual rules like $DD = D^2$ which is the same as saying
	\begin{equation*} 
		\dfrac{d}{dx}\left(\dfrac{d}{dx}(f)\right) = \dfrac{d^2f}{dx^2}.
	\end{equation*}
	We also have things like
	\begin{equation*} 
		(D + 1)^2 = D^2 + 2D + 1.
	\end{equation*}
	Note that the $1$ above is the operator $1,$ that is, $1f = f$ for any function $f.$\\
	(Don't make the mistake of thinking something like $D(D + 1) = D^2$ because $D1 = 0;$ the $1$ here is not the constant function.)\\
	This also shows how the original ODE (\ref{eq:const}) relates to the polynomial (\ref{eq:char}). To make it clearer, note that the ODE can simply be written as
	\begin{equation*} 
		(D^n + a_{n-1}D^{n-1} + \cdots + a_0)y = 0.
	\end{equation*}
\end{mdframed}
\newpage
\begin{mdframed}[style=boxstyle, frametitle={Annihilators of special functions}]
	From the discussion in the previous section, we can already see annihilators of some special functions as follows:
	\begin{center}
	\bgroup
	\def\arraystretch{1.25}
	\begin{tabular}{|l|l|}
		\hline
		Function & Annihilator\\
		\hline
		$e^{mx}$ & $D - m$\\
		$x^{k}e^{mx}$ & $(D - m)^{k+1}$\\
		$\sin bx, \cos bx$ & $D^2 + b^2$\\
		$x^k\sin bx, x^k\cos bx$ & $(D^2 + b^2)^{k+1}$\\
		$e^{ax}\sin bx, e^{ax}\cos bx$ & $(D - a)^2 + b^2$\\
		$x^ke^{ax}\sin bx, x^ke^{ax}\cos bx$ & $((D - a)^2 + b^2)^{k+1}$\\
		\hline
	\end{tabular}
	\egroup
	\end{center}
\end{mdframed}	
Note that all the annihilators in the above table are polynomials in $D,$ we will usually write an arbitrary such operator as $P(D).$\\
With the above things in mind, we proceed to the next subsection. 

\newpage

\subsection{The main problem}
\begin{mdframed}[style=boxstyle, frametitle={The Setup}]
	We consider a linear ODE of the form
	\begin{equation} \label{eq:constQ}
		(D^n + a_{n-1}D^{n-1} + \cdots + a_0)y = Q(x),
	\end{equation}
	where each $a_i$ is a constant and $Q(x)$ is one of the special functions listed in the table earlier.
\end{mdframed}
\begin{mdframed}[style=boxstyle, frametitle={The Solution}]
	We do this quite systematically.
	\begin{enumerate}[leftmargin=*, label = \Roman*.]
		\item First, we consider the associated homogeneous equation
		\begin{equation*} 
			(D^n + a_{n-1}D^{n-1} + \cdots + a_0)y = 0.
		\end{equation*}
		This can be solved completely by the methods we saw in \S\S\ref{ssec:undcoeffhomo}.\\
		Let the \emph{general} solution of this be $y_g(x).$
		\item As $Q(x)$ was a special function, we take its annihilator $P(D)$ from the table and apply it to both sides of (\ref{eq:constQ}). This gives us an equation of the form
		\begin{equation} \label{eq:temp1}
			P(D)(D^n + a_{n-1}D^{n-1} + \cdots + a_0)y = 0.
		\end{equation}
		Note that (\ref{eq:temp1}) is again a constant coefficients ODE (why? Recall the arithmetic.) and hence, we can solve it completely.\\
		Let this solution by $y_{g'}(x).$
		\item This $y_{g'}$ will be a sum of special functions. It will have all $n$ of the original special functions in $y_g$ and $k$ more. ($k = \deg P(D).$)\\
		Thus,
		\[y_{g'}(x) = y_g(x) + c_1y_1(x) + \cdots + c_ky_k(x),\]
		where the $y_i$s are the $k$ new functions. We now solve for \textbf{undetermined coefficients} $c_i$s by substituting the above solution back in (\ref{eq:constQ}). (Note that $y_g$ will get completely annihilated and can be ignored for better calculations.)\\
		Thus, once we solve for the $c_i$s, we are done and the final \emph{general} solution is
		\[y(x) = y_g(x) + c_1y_1(x) + \cdots + c_ky_k(x).\]
	\end{enumerate}
\end{mdframed}
\begin{mdframed}[style=boxstyle, frametitle={Slightly more general}]
	Note that the original restriction that $Q(x)$ be a special function was unnecessary. Indeed, we can do better and allow $Q(x)$ to be a \emph{linear combination} of the special functions.\\
	We can solve the smaller equations individually and finally add them (with appropriate scaling) to get the final solution.\\
	To be more explicit in terms of an example:\\
	Consider the ODE $(D^2 + 1)y = e^x + 2\sin x.$\\
	We shall first solve to get the general solution $y_g$ of $(D^2 + 1)y = 0.$\\
	Then, we get a particular solution $y_{p_1}$ of $(D^2 + 1)y = e^x$ using II and III from above.\\
	Similarly, we get a particular solution $y_{p_2}$ of $(D^2 + 1)y = \sin x.$\\ Then, the final solution is $y_g + y_{p_1} + 2y_{p_2}.$
\end{mdframed}
\begin{mdframed}[style=boxstyle2, frametitle={A note about calculations}]
	Sometimes, it may be computationally easier to \emph{not} break $Q(x)$ into all of its components.\\
	For example, if $Q(x) = e^x + xe^x + e^{2x},$ it would be better to break the problem into that for $(e^x + xe^x)$ and that for $e^{2x}.$\\
	The reason for this is that $(D - 1)^2$ is an annihilator for the former and would minimise repeated calculations.\\
	(Try some problems yourself, it's easier to just do it and realise what's best!)
\end{mdframed}