\sec{Cauchy Euler equations}
\subsection{The first challenge} \label{ssec:cauchyeulerhomo}
\begin{mdframed}[style=boxstyle, frametitle={The Setup}]
	An ODE of the form
	\begin{equation*} 
		x^ny^{(n)} + a_{n-1}x^{n-1}y^{(n-1)} + \cdots + a_{1}xy' + a_0y = 0
	\end{equation*}
	is called a \defin{Cauchy Euler equation}. (Each $a_i$ is a constant.)
\end{mdframed}
We shall be interested in solving the above equation only on $(0, \infty).$
\begin{mdframed}[style=boxstyle, frametitle={The Solution}]
	We plug the trial solution $y = x^m.$ This gives us the polynomial equation:
	\begin{align} 
		m(m - 1)\cdots(m-(n-1)) + a_{n-1}m(m-1)\cdots(m-(n-2)) + \nonumber \\
		 \cdots + a_1m + a_0 = 0. \label{eq:indic}
	\end{align}
	Note that if $m$ is a root of the above, then $x^m$ is a solution of the ODE.\\
	Thus, if the above equation has $n$ distinct roots $m_1, \ldots, m_n,$ then the original ODE has $n$ linearly independent solutions $x^{m_1}, \ldots, x^{m_n}$ and we are done.
\end{mdframed}
\begin{mdframed}[style=boxstyle, frametitle={Repeated roots}]
	Suppose $m_0$ is a repeated root of (\ref{eq:indic}). Suppose that it is repeated $k$ times. Then, one can show that the following $k$ functions are solutions of the original ODE:
	\begin{equation*} 
		x^{m_0}, x^{m_0}\ln x, \ldots, x^{m_0}(\ln x)^{k-1}.
	\end{equation*}
	These are linearly independent as well.\\
	Thus, we are now done for any case as the total number of roots is always going to be $n$ when counted with multiplicity. (This is assuming that we work in $\mathbb{C},$ which we shall do.)
\end{mdframed}
\begin{mdframed}[style=boxstyle, frametitle={Getting real}]
	Suppose $m = a+ib$ is a root. Then, we have
	\begin{align*} 
		x^m &:= \exp(m \ln x) & (\text{By definition})\\
		&= \exp(a \ln x + ib\ln x)\\
		&= \exp(a \ln x)\cdot\exp(ib \ln x)\\
		&= \exp(a \ln x)\cdot\big(\cos(b\ln x) + i\sin(b \ln x)\big)\\
		&=x^a(\cos(b\ln x) + i\sin(b \ln x))
	\end{align*}
	As before, if each $a_i$ is real, then the roots appear in the same multiplicity as their conjugates which will give us the real pair of solutions as:
	\begin{equation*} 
		x^a\cos(b\ln x), x^a\sin(b\ln x).
	\end{equation*}
	In case of repetition twice, we get the four solutions:
	\begin{equation*} 
		x^a\cos(b\ln x), x^a\sin(b\ln x),\quad x^a(\ln x)\cos(b\ln x), x^a(\ln x)\sin(b\ln x).
	\end{equation*}
	The general case is (hopefully) clear.
\end{mdframed}
\subsection{Welcome back, annihilators}
This time, we consider annihilators of polynomial (polylogmial?) functions.\\
As before, we shall see that it suffices to consider the case of just the monomials first. The annihilator table is particularly simple this time:
\begin{mdframed}[style=boxstyle, frametitle={Annihilators of special functions}]
	\begin{center}
	\bgroup
	\def\arraystretch{1.25}
	\begin{tabular}{|l|l|}
		\hline
		Function & Annihilator\\
		\hline
		$x^n$ & $xD - n$\\
		$x^n(\ln x)^k$ & $(xD - n)^{k+1}$\\
		\hline
	\end{tabular}
	\egroup
	\end{center}
\end{mdframed}
\begin{mdframed}[style=boxstyle, frametitle={A word on $xD$}]
	Note that $xD$ is the operator which acts on a function as:
	\begin{equation*} 
		(xD)(f) = xf'.
	\end{equation*}
	On the other hand, $Dx$ is an operator which acts as:
	\begin{equation*} 
		(Dx)(f) = D(xf) = xf' + f.
	\end{equation*}
	In particular, $xD \neq Dx.$ This means that $(xD)^2 \neq x^2D^2$ and so on.
\end{mdframed}
\emph{Note:} $xD$ is an operator but there is no $xP$ operator. \hfill (A joke.)
\begin{mdframed}[style=boxstyle, frametitle={Some arithmetic}]
	For ease of calculations, one may note the following useful identities`:
	\begin{align*} 
		x^2D^2 &= xD(xD - 1)\\
		x^3D^3 &= xD(xD - 1)(xD - 2)\\
		&\vdots
	\end{align*}
\end{mdframed}
\subsection{The main problem}
\begin{mdframed}[style=boxstyle, frametitle={The Setup}]
	We consider a linear ODE of the form
	\begin{equation} 
		x^ny^{(n)} + a_{n-1}x^{n-1}y^{(n-1)} + \cdots + a_{1}xy' + a_0y = Q(x),
	\end{equation}
	where each $a_i$ is a constant and $Q(x)$ is a linear combination of functions of the form $x^k(\ln x)^m$.
\end{mdframed}
\begin{mdframed}[style=boxstyle, frametitle={The Solution}]
	The steps are now identical as the case of \nameref{sec:undcoeff}.\\
	We shall assume that $Q(x) = x^k(\ln x)^m$ and not a linear combination. The general case follows as before by considering the linear combination of solutions.
	\begin{enumerate}[leftmargin=*, label = \Roman*.]
		\item We first consider the equation
		\begin{equation*} 
			x^ny^{(n)} + a_{n-1}x^{n-1}y^{(n-1)} + \cdots + a_{1}xy' + a_0y = 0.
		\end{equation*}
		This can be solved by the method we saw in \S\S\ref{ssec:cauchyeulerhomo}.
		\item We then apply the annihilator of $Q(x)$ from the table earlier.\\
		\textbf{\emph{This again gives us a Cauchy Euler equation.}}\\
		We solve this again using the method of \S\S\ref{ssec:cauchyeulerhomo}.
		\item We follow the step III as in the case of \nameref{sec:undcoeff} to determine the coefficients of the new functions we obtain in Step II.
	\end{enumerate}
\end{mdframed}
Note that $D^{11}$ is also an annihilator of $x^{10},$ however we do not use that.\\
The first reason is that the calculations would be a nightmare.\\
Secondly, applying this annihilator to both sides of the equation wouldn't technically give us a Cauchy Euler equation again. \\
Also, note that the method we have used doesn't require $k$ to be an integer. This would work even if we wish to solve something like $xy'' + y = x^{1/2}.$ Another reason to prefer the annihilator $xD - 1/2.$